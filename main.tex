\documentclass{article}
\usepackage{amsmath}
\usepackage{amsfonts}
\usepackage[russian]{babel}
\usepackage[left=2cm,right=2cm,
top=2cm,bottom=2cm,bindingoffset=0cm]{geometry}
\usepackage[utf8]{inputenc}
\renewcommand{\baselinestretch}{1.5}

\title{Все обо всем}
\author{Винницкий Всеволод}
\date{October 2022}

\begin{document}

\maketitle

\section{Git}
\subsection{Работа с GitHub}
Чтобы залить pull-request в репозитарий с GitHub (и это репозитарий не ваш!) то для начала, этот репозитарий необходимо склонировать себе: например нажать fork и в своем акаунте GitHub появится копия репозитария. (Если у пользователя CustomI был репозитарий rep, то после форка у нас появится репозитарий с именем My\_name/rep). \\
После того как репозитарий форкнут, можно его склонировать себе на машину. Склонировать можно двумя способами: по HTTPS или по SSH. От этих двух подходов зависит по какому протоколу будут данные "передаваться" с нашей локальной машины в удаленный репозитарий на GitHub.
Команды для клонирования:
\begin{itemize}
\item SSH : git@github.com:My\_name/rep.git
\item HTTPS : https://github.com/My\_name/rep.git
\end{itemize}

Эти команды можно не запоминать, а посмотреть в браузере нажав в репозитарии на кнопку code. Клонирование осуществляется командой:
\\
\\
\fbox {git clone COMMAND}
\\
\\
COMMAND - это одна из команд для SSH/HTTPS. Теперь репозитарий есть у нас локально!
\\
Как запушить на GitHub?
По SSH (эти действия нужно делать один раз для пары (аккаунт на GitHub, компьютер)). 

\begin{itemize}
\item Генерируем два ключа: открытый и секретный. Команда:
\\
\fbox {ssh-keygen -t rsa -b 4096 -C <<your\_email@example.com>>}
\item далее жмем enter, все значения будут проставлены по дефолту. После чего в каталоге /Users/you/.ssh/id\_rsa будет создано да файла: id\_rsa id\_rsa.pub (собственно секретный и публичный ключ).

\item Далее запускаем SSH-агента:
\\
\fbox {eval <<\$(ssh-agent -s)>>}
\\
и добавляем в него ключ командой:
\\
\fbox {ssh-add ~/.ssh/id\_rsa}
\item Осталось добавить публичный ключ в свой аккаунт на GitHub: для этого из аккаунта переходим в settings, далее в SSH and GPG keys. Добавляем публичный ключ.
\end{itemize}

Теперь можно делать пуш: 
\fbox {git push git@github.com:My\_name/rep.git}

\end{document}